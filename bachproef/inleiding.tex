%%=============================================================================
%% Inleiding
%%=============================================================================

\chapter{\IfLanguageName{dutch}{Inleiding}{Introduction}}
\label{ch:inleiding}

Videospellen zijn de dag van vandaag niet meer weg te denken uit onze samenleving. Tot begin jaren 2000 werden ze vooral gespeeld op computers of gespecialiseerde systemen door de meer gepassioneerde gebruiker. Geholpen door het wijdverspreide gebruik van smartphones en tablets heeft de sector een demografische verschuiving gekend, richting de zogenaamde ``casual games''. Deze zijn vooral gericht op het bredere publiek, dit in tegenstelling tot de eerdere spellen die vooral gericht waren op het nichepubliek.

De gaming-industrie heeft, mede dankzij deze massale adoptie van mobiele systemen, een explosieve groei gekend. Dit is in 2020 nogmaals bewezen dankzij een recordomzet van maar liefst \$159,3 miljard dollar. In de nabije toekomst ziet het er niet naar uit dat deze groei zal stoppen. Verwacht wordt dat de omzet nog verder zal stijgen tot \$200,8 miljard dollar in 2023 \autocite{WePC2021}.

Games worden ontworpen met de primaire gedachtegang dat de gebruiker vermaakt moet worden. Ze kunnen de wenselijke ervaring creëren om gebruikers intens betrokken te blijven houden bij een activiteit gedurende lange perioden. Het is dan ook niet meer dan logisch dat ontwikkelaars en designers geïnteresseerd zijn om dit soort van gedrag te stimuleren buiten de gamingwereld om. Het is daarom dat ze kenmerken uit het game-design veld proberen toe te passen om hun niet-game gerelateerde producten, diensten of toepassingen aangenamer en boeiender te maken. Dit concept is gekend als gamification.

Eén van de meeste bekende en succesvolle platformen dat gebruik maakt van gamification is \textit{Foursquare}\footnote{https://foursquare.com/}, door hun gebruik van punten en badges. Mede dankzij dit succes heeft gamification enorm aan terrein gewonnen. Verschillende bedrijven bieden nu zelfs gamification aan als een softwaredienstenlaag.

\section{\IfLanguageName{dutch}{Probleemstelling}{Problem Statement}}
\label{sec:probleemstelling}

\section{\IfLanguageName{dutch}{Onderzoeksvraag}{Research question}}
\label{sec:onderzoeksvraag}

Dit onderzoek zal zich focussen op het implementeren van gamification en de effecten hiervan op de gebruikersinteractie en -retentie. Om hierop een zo uitgebreid mogelijk antwoord te geven werden de volgende onderzoeksvragen opgesteld:

\begin{itemize}
 \item Op welke verschillende manieren kan gamification worden geïmplementeerd?
 \item Welke stappen zijn nodig om gamification toe te voegen aan een reeds bestaand platform?
 \item Vergroot het toevoegen van gamification de gebruikersinteractie en -retentie?
\end{itemize}

\section{\IfLanguageName{dutch}{Onderzoeksdoelstelling}{Research objective}}
\label{sec:onderzoeksdoelstelling}

Het voornaamste doel van dit onderzoek is om een prototype te ontwikkelen waarin verschillende elementen uit het game-design veld worden geïmplementeerd. Dit prototype zal als basis dienen om een gebruikersonderzoek uit te voeren waarbij wordt nagegaan wat het effect is van de ontwerpelementen op de gebruikersinteractie en -retentie.

\section{\IfLanguageName{dutch}{Opzet van deze bachelorproef}{Structure of this bachelor thesis}}
\label{sec:opzet-bachelorproef}

% Het is gebruikelijk aan het einde van de inleiding een overzicht te
% geven van de opbouw van de rest van de tekst. Deze sectie bevat al een aanzet
% die je kan aanvullen/aanpassen in functie van je eigen tekst.

De rest van deze bachelorproef is als volgt opgebouwd:

In Hoofdstuk~\ref{ch:stand-van-zaken} wordt een overzicht gegeven van de stand van zaken binnen het onderzoeksdomein, op basis van een literatuurstudie.

In Hoofdstuk~\ref{ch:methodologie} wordt de methodologie toegelicht en worden de gebruikte onderzoekstechnieken besproken om een antwoord te kunnen formuleren op de onderzoeksvragen.

In Hoofdstuk~\ref{ch:prototype} wordt een prototype toegelicht waarbij verschillende game-design elementen gebruikt worden om gamification te implementeren.

Vervolgens wordt in Hoofdstuk~\ref{ch:gebruikersonderzoek} het gebruikersonderzoek gevoerd. De bekomen resultaten worden geanalyseerd en besproken.

In Hoofdstuk~\ref{ch:conclusie}, tenslotte, wordt de conclusie gegeven en een antwoord geformuleerd op de onderzoeksvragen. Daarbij wordt ook een aanzet gegeven voor toekomstig onderzoek binnen dit domein.