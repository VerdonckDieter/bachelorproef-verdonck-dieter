%%=============================================================================
%% Conclusie
%%=============================================================================

\chapter{Conclusie}
\label{ch:conclusie}

% TODO: Trek een duidelijke conclusie, in de vorm van een antwoord op de
% onderzoeksvra(a)g(en). Wat was jouw bijdrage aan het onderzoeksdomein en
% hoe biedt dit meerwaarde aan het vakgebied/doelgroep? 
% Reflecteer kritisch over het resultaat. In Engelse teksten wordt deze sectie
% ``Discussion'' genoemd. Had je deze uitkomst verwacht? Zijn er zaken die nog
% niet duidelijk zijn?
% Heeft het onderzoek geleid tot nieuwe vragen die uitnodigen tot verder 
%onderzoek?

Tijdens dit onderzoek werd onderzocht hoe gamification kan worden geïmplementeerd en wat het effect hiervan is op de gebruikersinteractie en -retentie. Om dit te onderzoeken werd op de volgende onderzoeksvragen een antwoord gegeven:

\begin{itemize}
    \item Op welke verschillende manieren kan gamification worden geïmplementeerd?
    \item Welke stappen zijn nodig om gamification toe te voegen aan een reeds bestaand platform?
    \item Vergroot het toevoegen van gamification de gebruikersinteractie en -retentie?
\end{itemize}

Hiervoor werd een prototype uitgewerkt waarbinnen enkele ontwerpelementen van gamification werden toegevoegd. Op basis van dit prototype werd tevens een gebruikersonderzoek gevoerd.

Om de eerste onderzoeksvraag te kunnen beantwoorden werd onderzoek gedaan naar de verschillende ontwerpelementen binnen gamification. Dit deel van het onderzoek werd uitgevoerd in de literatuurstudie in Hoofdstuk~\ref{ch:stand-van-zaken}. Hieruit werd duidelijk dat gamification een aantal veelvoorkomende ontwerpelementen omvat waaronder: punten, badges, scoreborden, prestatiegrafieken, betekenisvolle verhalen, avatars en teamleden. Deze elementen worden het vaakst geïmplementeerd omdat ze direct zichtbaar zijn voor de spelers, gemakkelijk geactiveerd of gedeactiveerd worden en omdat ze spelers sterk motiveren. Voor de spelontwerpers zijn ze gemakkelijk te implementeren omdat ze deel uitmaken van het zichtbare deel en niet afhankelijk zijn van onderliggende mechanismen.

De tweede onderzoeksvraag onderzocht welke stappen nodig zijn om gamification toe te voegen aan een reeds bestaand platform. Zoals eerder vermeld werd een prototype uitgewerkt, op basis van het platform Innerdreams, waaraan een aantal ontwerpelementen werden toegevoegd: punten, badges, een scorebord en een beloningswinkel. Na het uitwerken van het prototype is duidelijk geworden dat het toevoegen van deze ontwerpelementen geen fundamentele veranderingen vereisten aan het bestaande systeem. De enige benodigde verandering was het aanpassen van de bestaande gebruikerstabel zodat deze de verzamelde punten kon bijhouden. Alle overige toevoegingen werden ingewerkt binnen het bestaande systeem zonder aan de reeds bestaande functionaliteiten te raken.

De derde en laatste onderzoekvraag onderzocht of het toevoegen van gamification de gebruikersinteractie en -retentie vergroot. Zoals reeds vermeld werd op basis van het prototype een gebruikersonderzoek gevoerd. Uit de resultaten van dit onderzoek blijkt dat de gebruikersinteractie en -retentie wel degelijk vergroot wordt maar dat dit sterk afhankelijk is van het gebruikerstype. Het is daarom dat het noodzakelijk is om een gebruikersonderzoek te voeren als men ervoor kiest om gamification te gaan implementeren. Zo een gebruikersonderzoek is nodig om te kunnen bepalen uit welke soorten van gebruikerstypes de huidige of toekomstige gebruikers bestaan. Het is bij dit onderzoek ook belangrijk om te onthouden dat een gebruiker vaak niet één gebruikerstype heeft maar door meerdere types kan beschreven worden. Op basis van dit gebruikersonderzoek kan dan bepaald worden welke ontwerpelementen geïmplementeerd moeten worden en op welke manier ze moeten worden aangepast.

Aangezien het gebruikersonderzoek werd uitvoerd op een kleine steekproefgrootte (N = 66) kon niet worden bepaald of een associatie bestond tussen de verschillende gebruikerstypes en het geslacht of de leeftijden. Hierdoor werd bij het onderzoeken of gamification de gebruikersinteractie en -retentie verbeterd geen verder onderscheid gemaakt tussen geslacht of leeftijd. Een soortgelijk onderzoek met een grotere steekproefgrootte en een gelijkere verdeling zou hierover duidelijkheid kunnen scheppen.


