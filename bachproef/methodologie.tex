%%=============================================================================
%% Methodologie
%%=============================================================================

\chapter{\IfLanguageName{dutch}{Methodologie}{Methodology}}
\label{ch:methodologie}

%% TODO: Hoe ben je te werk gegaan? Verdeel je onderzoek in grote fasen, en
%% licht in elke fase toe welke stappen je gevolgd hebt. Verantwoord waarom je
%% op deze manier te werk gegaan bent. Je moet kunnen aantonen dat je de best
%% mogelijke manier toegepast hebt om een antwoord te vinden op de
%% onderzoeksvraag.

In dit hoofdstuk wordt de manier waarop dit onderzoek is uitgevoerd uitvoerig besproken. Het onderzoek werd gestart met een uitgebreide literatuurstudie, die kan teruggevonden worden in Hoofdstuk 2. In deze literatuurstudie werd gamification gedefinieerd en werden de verchillende ontwerpelementen in detail besproken. Ook werd de manier waarop gamification gedrag kan beïnvloeden bekeken.

Daarna zal de uitwerking van het uitgewerkte prototype in detail worden bekeken. Vervolgens zal het gebruikersonderzoek worden besproken en wordt de verzamelde data verwerkt.

\section{Prototype}

Om het effect van gamification op de gebruikersinteractie en -retentie te onderzoeken werd een prototype uitgewerkt op basis van een bestaand platform, Innerdreams\footnote{https://www.innerdreams.eu/nl-be/} genaamd. Dit platform is een enquêteplatform met als doel de mening in kaart te brengen van de moderne student en young starters. Dit prototype werd ontwikkeld binnen DNN\footnote{https://www.dnnsoftware.com/}, voorheen DotNetNuke. Het is een opensource contentmanagementsysteem voor het ASP.NET-framework, geschreven in C\#. DNN is volledig uitbreidbaar en aanpasbaar met behulp van skins, modules en providers.

Een aantal ontwerpelementen, die besproken zijn in de literatuurstudie, werden geïmplementeerd met als beoogde doel het platform interactiever te maken. Deze elementen zijn punten, badges en een scorebord. Ook werd gekozen om een beloningswinkel te implementeren, dit omdat het werkt als een extrinsieke motivator.

\section{Gebruikersonderzoek}

\section{Gegevensverwerking}
