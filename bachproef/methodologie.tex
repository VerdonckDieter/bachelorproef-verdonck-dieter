%%=============================================================================
%% Methodologie
%%=============================================================================

\chapter{\IfLanguageName{dutch}{Methodologie}{Methodology}}
\label{ch:methodologie}

%% TODO: Hoe ben je te werk gegaan? Verdeel je onderzoek in grote fasen, en
%% licht in elke fase toe welke stappen je gevolgd hebt. Verantwoord waarom je
%% op deze manier te werk gegaan bent. Je moet kunnen aantonen dat je de best
%% mogelijke manier toegepast hebt om een antwoord te vinden op de
%% onderzoeksvraag.

\section{Inleiding}

In dit hoofdstuk zal de manier waarop dit onderzoek is uitgevoerd uitvoerig worden besproken. Het onderzoek werd gestart met een uitgebreide literatuurstudie, die kan teruggevonden worden in Hoofdstuk~\ref{ch:stand-van-zaken}. In deze literatuurstudie werd gamification gedefinieerd en werden de verschillende ontwerpelementen in detail besproken. Ook werd de manier waarop gamification gedrag kan beïnvloeden bekeken. Hierna werd het prototype uitgewerkt waarbinnen een aantal van deze ontwerpelementen werden geïmplementeerd. Vervolgens werd een gebruikersonderzoek gevoerd en werden de resultaten hiervan geanalyseerd en besproken. Ten slotte werd op basis van deze resultaten een conclusie genomen.

\section{Prototype}

Om het effect van gamification op de gebruikersinteractie en -retentie te onderzoeken werd eerst een prototype uitgewerkt. Dit prototype werd ontwikkeld op basis van een bestaand platform, Innerdreams\footnote{https://www.innerdreams.eu/nl-be/} genaamd. Dit platform is een enquêteplatform met als doel de mening in kaart te brengen van de moderne student en young starters. De ontwikkeling van het prototype gebeurde binnen DNN\footnote{https://www.dnnsoftware.com/}, voorheen DotNetNuke. Het is een opensource contentmanagementsysteem voor het ASP.NET-framework, geschreven in C\#.

Een aantal ontwerpelementen, die besproken zijn in de literatuurstudie, werden geïmplementeerd met als beoogde doel het platform interactiever te maken. De gekozen elementen waren punten, badges en een scorebord. Ook werd gekozen om een beloningswinkel te implementeren waar de behaalde punten kunnen worden gespendeerd.

\section{Gebruikersonderzoek}

Dit onderzoek is gebaseerd op een tweeledig proces. Ten eerste werd gekozen om gebruikers het prototype van gamification binnen Innerdreams te laten gebruiken. Vervolgens werd op basis van deze ervaring een enquête afgenomen. Aan de hand van dit proces werd een antwoord gegeven op de laatste onderzoeksvraag.

\subsection{Gebruik van het prototype}

Om het gebruik van het prototype tot een goed einde te brengen werd aan de gebruikers een kort stappenplan meegegeven waarin uitleg werd gegeven over hoe het platform gebruikt wordt. Aan de deelnemers werd gevraagd om een tijdelijk gebruikrsprofiel aan te maken en hierna een testenquête af te leggen. Dit werd gedaan om het verkrijgen van punten duidelijk te maken. Tijdens het afleggen van deze testenquête kregen de gebruikers hun verzamelde punten te zien. Bij het succesvol voltooien van de enquête kregen ze een badge als visuele voorstelling van hun prestatie. Ook werd op het einde van de enquête een overzichtspagina weergegeven waarop de gebruikers hun verkregen punten te zien kregen en de mogelijkheid kregen om de enquête te delen, dit om nog extra punten te behalen. Na het voltooien konden de gebruikers het scorebord raadplegen om hun rangschikking, op basis van het totale aantal verzamelde punten, in vergelijking met alle andere gebruikers te zien te krijgen. Ten slotte kregen de gebruikers ook de mogelijkheid om de beloningswinkel te bekijken waarin een aantal beloningen te zien waren met hun respectievelijke puntenkost.

\subsection{Enquête}

Na het gebruik van het prototype van gamification binnen Innerdreams werd een online enquête afgenomen waarmee data kon verzameld worden over een breed publiek. De enquête werd opgesteld via Google Forms. Hiermee kon snel een relatief grote enquête worden opgesteld en kon een snelle data-analyse worden uitgevoerd. In deze enquête werd aan de deelnemers gevraagd om 33 vragen te beantwoorden, bestaande uit de volgende drie delen:

\begin{itemize}
    \item Demografie: 2 vragen over de samenstelling van de groep van deelnemers met als doel de steekproef te gaan beschrijven en om te controleren of deze representatief is.
    \item Hexad Schaal: 24 vragen op een Likert-schaal van 7 punten op basis van de vragen uit de Gamification User Types Hexad Scale. Deze vragen hadden als doel de deelnemers onder te verdelen gebaseerd op hun voorkeuren wanneer ze met elementen van gamification omgaan.
    \item Innerdreams: 7 stellingen op een Likert-schaal van 5 punten. Deze stellingen werden beantwoord op basis van het gebruik van het prototype van gamification binnen Innerdreams. Het doel van deze stellingen was om na te gaan of de gebruikersinteractie en -retentie wel degelijk verbeterd is.
\end{itemize}
