%%=============================================================================
%% Voorwoord
%%=============================================================================

\chapter*{\IfLanguageName{dutch}{Woord vooraf}{Preface}}
\label{ch:voorwoord}

%% TODO:
%% Het voorwoord is het enige deel van de bachelorproef waar je vanuit je
%% eigen standpunt (``ik-vorm'') mag schrijven. Je kan hier bv. motiveren
%% waarom jij het onderwerp wil bespreken.
%% Vergeet ook niet te bedanken wie je geholpen/gesteund/... heeft

Deze bachelorproef werd geschreven in functie van het succesvol afronden van de opleiding Bachelor in de Toegepaste Informatica, met als afstudeerrichting Mobile Applications.

Tijdens mijn zoektocht naar een stage kwam ik een vacature van Guido NV tegen waarin vermeld werd dat een van de opdrachten ‘implementatie van gamification in een bestaand platform’ omvatte. Dit onderwerp sprak mij direct aan en motiveerde mij om op de vacature te reageren gezien gamification mij niet bekend was en ik graag kennismaak met nieuwe onderwerpen. Na te onderzoeken wat gamification precies omvatte had ik dan ook besloten om een onderzoek te doen rond gamification.

Deze bachelorproef zou niet tot stand zijn gekomen zonder de hulp van verscheidene mensen. Ik zou hun dan ook nog eens extra willen bedanken.

Ten eerste zou ik mijn co-promotor, Gunther Christiaens, willen bedanken voor alle ondersteuning tijdens zowel de uitvoering van de stage als de bachelorproef. Ik kon steeds rekenen op een snel antwoord als ik met een vraag zat of problemen had.

Ik wil zeker ook mijn promotor, Pieter Van Der Helst, bedanken om mij initeel op weg te helpen bij het uitvoeren van dit onderzoek en voor de feedback tijdens het schrijven van de bachelorproef.

Ook zou ik Sharon Van Hove willen bedanken, die als vervangende promotor optrad. Mevrouw van Hove heeft mij waardevolle feedback gegeven waardoor ik dit onderzoek tot een goed einde kon brengen.


