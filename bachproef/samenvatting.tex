%%=============================================================================
%% Samenvatting
%%=============================================================================

% TODO: De "abstract" of samenvatting is een kernachtige (~ 1 blz. voor een
% thesis) synthese van het document.
%
% Deze aspecten moeten zeker aan bod komen:
% - Context: waarom is dit werk belangrijk?
% - Nood: waarom moest dit onderzocht worden?
% - Taak: wat heb je precies gedaan?
% - Object: wat staat in dit document geschreven?
% - Resultaat: wat was het resultaat?
% - Conclusie: wat is/zijn de belangrijkste conclusie(s)?
% - Perspectief: blijven er nog vragen open die in de toekomst nog kunnen
%    onderzocht worden? Wat is een mogelijk vervolg voor jouw onderzoek?
%
% LET OP! Een samenvatting is GEEN voorwoord!

%%---------- Nederlandse samenvatting -----------------------------------------
%
% TODO: Als je je bachelorproef in het Engels schrijft, moet je eerst een
% Nederlandse samenvatting invoegen. Haal daarvoor onderstaande code uit
% commentaar.
% Wie zijn bachelorproef in het Nederlands schrijft, kan dit negeren, de inhoud
% wordt niet in het document ingevoegd.

\IfLanguageName{english}{%
\selectlanguage{dutch}
\chapter*{Samenvatting}
\lipsum[1-4]
\selectlanguage{english}
}{}

%%---------- Samenvatting -----------------------------------------------------
% De samenvatting in de hoofdtaal van het document

\chapter*{\IfLanguageName{dutch}{Samenvatting}{Abstract}}

Deze bachelorproef werd gemaakt met als doel te onderzoeken op welke verschillende manieren gamification kan geïmplementeerd worden, welke stappen hiervoor nodig zijn en wat het effect is van gamification op de gebruikersinteractie en -retentie. Gamification heeft als term de laatste tien jaar sterk aan populariteit gewonnen en wordt alsmaar populairder aangezien het het gebruikersengagement kan vergroten \autocite{Deterding20112}.

Allereerst werd een grondig literatuuronderzoek uitgevoerd over de verschillende ontwerpelementen binnen gamification en wat het effect hiervan is op het menselijke gedrag. Vervolgens werd een prototype uitgewerkt waarin enkele van deze ontwerpelementen werden geïmplementeerd binnen een bestaand platform. Om het effect van gamification op de gebruikersinteractie en -retentie te onderzoeken werd vervolgens een gebruikersonderzoek gevoerd bestaande uit twee delen. In het eerste deel van het gebruikersonderzoek werd het uitgewerkte prototype gebruikt door een groep van deelnemers om de werking hiervan te valideren en om de deelnemers kennis te laten maken met gamification. Hierna werd in het tweede deel van het onderzoek een enquête afgenomen met als doel demografische gegevens over de deelnemers te verzamelen en de deelnemers onder te verdelen in verschillende gebruikerstypes. Ook werd, op basis van deze onderverdeling, het effect van de ontwerpelementen uit het prototype onderzocht.

Uiteindelijk kon besloten worden dat gamification een enorme waaier aan ontwerpelementen bevat. Het implementeren van een aantal van de meest voorkomende elementen toont ook aan dat niet aan de bestaande functionaliteiten moet worden geraakt om gamification te implementeren. Ten slotte kon ook besloten worden dat gamification de gebruikersinteractie en -retentie wel degelijk kan verbeteren maar dat dit sterk afhankelijk is van de soort gebruiker. Het is belangrijk dat alvorens gamification wordt geïmplementeerd eerst een gebruikersonderzoek wordt gevoerd om te gaan bepalen welke ontwerpelementen nodig zijn en hoe ze moeten aangepast worden.
